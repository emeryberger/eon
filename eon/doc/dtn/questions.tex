 \tolerance = 1000
\documentclass[onecolumn, 10pt]{article}
\usepackage{epsfig}
\usepackage{amssymb, amsmath}
%\usepackage[pdftex]{graphicx}
\usepackage{graphicx}


 \begin{document}
 Routing for Turtlenet
 
 Constraints:
 \begin{itemize}
 \item Buffer
 \item Energy
 \end{itemize}
 
 
 Some questions to help design \begin{itemize} 

\item is it possible to
 get instant feedback when communicating with turtle? i.e., does a
 turtle know if its peer has received a packet. {\bf yes}

 \item Does meeting history mean anything? Do turtles tend to visit
 same places as they usually do. {\bf not sure, but probably}

 \item What is the rate of data generation. How long does it take for
 buffer to fill up? {\bf Maximum of 10 bytes an hour. With 128 KB
 available, it will take about 76 weeks for the data to be
 over-written}.

 \item Can we always assume that an older packet will be delivered
 before a new packet. Is this even needed? {\bf This will be difficult
 if we assume packets are being replicated. Not difficult if we assume
 the same set of packets are replicated on all links.  This might end
 up being very inefficient, though.}
 
 \item Can we assume that when meeting the access point, all packets
 are delivered? {\bf No, Because Receive energy is less than
 transmission energy, it's a better idea to receive acks from access
 point and then deliver only relevant packets.  This is probably how
 all connection events happen.}.
 

\item Is there a difference between transmission energy and receive
 energy?{\bf Yes, transmission energy is about 2 to 4 times greater
 than receive} 

 \item How power-constrained are the access points?{\bf not nearly as
 much as the turtle devices.  They will never go under water.}
 \end{itemize}
 
 Routing questions
 \begin{itemize}

 \item How to choose what packets to keep and what to delete?{\bf
 Priority based on time, priority to turtle which has the most
 information that has not reached the access point, message-based:
 e.g.more priority to GPS readings than o temperature readings} 

 \item Is deletion like a queue operation or can packets in the center
 be deleted?{\bf deletion can be performed ONLY from either end}

 \item How to use history information {\bf ?}

\end{itemize}

Some ideas
\begin{itemize}
 \item Partition storage into several blocks and assign each block for
 a particular type of data (packets generated by the turtle, packets
 replicas from other turtles and acks). The size of the block will
 automatically control replication, acknowledgment deletion etc.

 \item Use ack-ranges instead of individual acks. For e.g., all
 packets generated within time t1-t2 for {\em turtle~A} have been
 delivered. This will reduce the number of acks and make it easier to
 delete packets from one end

 \item Use acks to determine which turtle does not have any packets
 delivered, and give higher priority to packets from that turtle

 \item it might be simpler to allow replication only from the source
 turtle and not from others. Though i am not sure if it will be
 effective

\end{itemize}

Variables that can be adapted:
\begin{itemize}
 \item Transmission power level 
 \item Buffer size for different types of data
 \item How much to replicate? Stop after $k$ replicas? 
 \item Does new data always overwrite old data or can we choose not to overwrite and delete the new data instead?
 \end{itemize}

\end{document}
